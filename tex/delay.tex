\documentclass[letterpaper]{article}
\usepackage{natbib,alifexi}
\usepackage[utf8]{inputenc}

\title{Estimation en temps réel des retards dans un réseau\\ de bus à l'aide de données historiques}
\author{Nikita Marchant$^{1}$\\
\mbox{}\\
$^1$Université Libre de Bruxelles, Département d'Informatique\\
nimarcha@ulb.ac.be}


\begin{document}
\maketitle

\begin{abstract}

\end{abstract}

\section{Introduction}



\section{K plus proches voisins}

L'algorithme des K plus proches voisins (ou k-nearest neighbors ou encore k-NN) est une méthode d'intelligence articifielle \citep{trevor2009elements}


\begin{eqnarray}
v\sim\sqrt{D\Delta\epsilon}\;, \label{eq4}
\end{eqnarray}


\footnotesize
\bibliographystyle{apalike}
\bibliography{example}


\end{document}
